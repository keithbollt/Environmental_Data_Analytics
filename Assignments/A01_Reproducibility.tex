\documentclass[]{article}
\usepackage{lmodern}
\usepackage{amssymb,amsmath}
\usepackage{ifxetex,ifluatex}
\usepackage{fixltx2e} % provides \textsubscript
\ifnum 0\ifxetex 1\fi\ifluatex 1\fi=0 % if pdftex
  \usepackage[T1]{fontenc}
  \usepackage[utf8]{inputenc}
\else % if luatex or xelatex
  \ifxetex
    \usepackage{mathspec}
  \else
    \usepackage{fontspec}
  \fi
  \defaultfontfeatures{Ligatures=TeX,Scale=MatchLowercase}
\fi
% use upquote if available, for straight quotes in verbatim environments
\IfFileExists{upquote.sty}{\usepackage{upquote}}{}
% use microtype if available
\IfFileExists{microtype.sty}{%
\usepackage{microtype}
\UseMicrotypeSet[protrusion]{basicmath} % disable protrusion for tt fonts
}{}
\usepackage[margin=2.54cm]{geometry}
\usepackage{hyperref}
\hypersetup{unicode=true,
            pdftitle={Assignment 1: Reproducibility, Workflow, Version Control},
            pdfauthor={Keith Bollt},
            pdfborder={0 0 0},
            breaklinks=true}
\urlstyle{same}  % don't use monospace font for urls
\usepackage{graphicx,grffile}
\makeatletter
\def\maxwidth{\ifdim\Gin@nat@width>\linewidth\linewidth\else\Gin@nat@width\fi}
\def\maxheight{\ifdim\Gin@nat@height>\textheight\textheight\else\Gin@nat@height\fi}
\makeatother
% Scale images if necessary, so that they will not overflow the page
% margins by default, and it is still possible to overwrite the defaults
% using explicit options in \includegraphics[width, height, ...]{}
\setkeys{Gin}{width=\maxwidth,height=\maxheight,keepaspectratio}
\IfFileExists{parskip.sty}{%
\usepackage{parskip}
}{% else
\setlength{\parindent}{0pt}
\setlength{\parskip}{6pt plus 2pt minus 1pt}
}
\setlength{\emergencystretch}{3em}  % prevent overfull lines
\providecommand{\tightlist}{%
  \setlength{\itemsep}{0pt}\setlength{\parskip}{0pt}}
\setcounter{secnumdepth}{0}
% Redefines (sub)paragraphs to behave more like sections
\ifx\paragraph\undefined\else
\let\oldparagraph\paragraph
\renewcommand{\paragraph}[1]{\oldparagraph{#1}\mbox{}}
\fi
\ifx\subparagraph\undefined\else
\let\oldsubparagraph\subparagraph
\renewcommand{\subparagraph}[1]{\oldsubparagraph{#1}\mbox{}}
\fi

%%% Use protect on footnotes to avoid problems with footnotes in titles
\let\rmarkdownfootnote\footnote%
\def\footnote{\protect\rmarkdownfootnote}

%%% Change title format to be more compact
\usepackage{titling}

% Create subtitle command for use in maketitle
\newcommand{\subtitle}[1]{
  \posttitle{
    \begin{center}\large#1\end{center}
    }
}

\setlength{\droptitle}{-2em}

  \title{Assignment 1: Reproducibility, Workflow, Version Control}
    \pretitle{\vspace{\droptitle}\centering\huge}
  \posttitle{\par}
    \author{Keith Bollt}
    \preauthor{\centering\large\emph}
  \postauthor{\par}
    \date{}
    \predate{}\postdate{}
  

\begin{document}
\maketitle

\subsection{OVERVIEW}\label{overview}

This exercise accompanies the lessons in Environmental Data Analytics
(ENV872L) on reproducibility, workflow, and version control.

\subsection{Directions}\label{directions}

\begin{enumerate}
\def\labelenumi{\arabic{enumi}.}
\tightlist
\item
  Change ``Student Name'' on line 3 (above) with your name.
\item
  Use the lesson as a guide. It contains code that can be modified to
  complete the assignment.
\item
  Work through the steps, \textbf{creating code and output} that fulfill
  each instruction.
\item
  Be sure to \textbf{answer the questions} in this assignment document.
  Space for your answers is provided in this document and is indicated
  by the ``\textgreater{}'' character. If you need a second paragraph be
  sure to start the first line with ``\textgreater{}''. You should
  notice that the answer is highlighted in green by RStudio.
\item
  When you have completed the assignment, \textbf{Knit} the text and
  code into a single PDF file. You will need to have the correct
  software installed to do this (see Software Installation Guide) Press
  the \texttt{Knit} button in the RStudio scripting panel. This will
  save the PDF output in your Assignments folder.
\item
  After Knitting, please submit the completed exercise (PDF file) to the
  dropbox in Sakai. Please add your last name into the file name (e.g.,
  ``Salk\_A01\_Reproducibility.pdf'') prior to submission.
\end{enumerate}

The completed exercise is due on Thursday, 17 January, 2018 before class
begins.

\subsection{1) Discussion Questions}\label{discussion-questions}

\subsubsection{Question}\label{question}

Why are reproducible practices becoming the norm in data analytics?

\begin{quote}
Answer: They allow people to share the analysis they have been working
on with both project members and the broader community. They also allow
their work to be, well, reproduced and built off of by others.
\end{quote}

\subsubsection{Question}\label{question-1}

What are your previous experiences with data analytics, R, and Git?
Include both formal and informal training.

\begin{quote}
Answer: I have never used R or Git. I did some statistical analysis in
several undergrad classes, including working with some biology software
in a population dynamics class. I took Intro to GIS last semester, which
used some Python I believe.
\end{quote}

\subsubsection{Question}\label{question-2}

Are there any components of the course about which you feel confident?

\begin{quote}
Answer: I feel pretty good about the statistical analysis aspect of the
course because I have some experience with basic statistical analyses
from previous classes.
\end{quote}

\subsubsection{Question}\label{question-3}

Are there any components of the course about which you feel
apprehensive?

\begin{quote}
Answer: Well, yes. I have never explicitly coded before, and that is my
number one concern. I dislike the feeling of being stuck, which I
understand happens a lot in data analysis and in coding more broadly.
\end{quote}

\subsection{2) GitHub}\label{github}

\subsubsection{Your Repository}\label{your-repository}

Provide a link below to your course repository in GitHub. Make sure you
have pulled all recent changes from the course repository
(\url{https://github.com/KateriSalk/Environmental_Data_Analytics}) and
that you have updated your course README file.

\begin{quote}
Answer: \url{https://github.com/KateriSalk/Environmental_Data_Analytics}
\end{quote}


\end{document}
